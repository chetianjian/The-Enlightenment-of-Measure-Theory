\documentclass[reqno]{amsart}
\usepackage{lipsum}
\usepackage{amsfonts}
\usepackage{amsmath}
\usepackage{amssymb}

\theoremstyle{definition}
\newtheorem{theorem}{Theorem}
\newtheorem{lemma}{Lemma}
\newtheorem{definition}{Definition}
\newtheorem{notation}{Notation}
\newtheorem{problem}{Problem}
\newtheorem{remark}{Remark}{\normalfont}
\newtheorem{proposition}{Proposition}
\newtheorem{corollary}{Corollary}
\newtheorem{example}{Example}

\begin{document}
\author{Tianjian Che}
\title{Proof to Theorems Related to Lebesgue Measure}
\date{\today}
\maketitle

\section{Why introducing measure - The deficiency of Riemann Integral}
We introduced measure and Lebesgue Integral due to the deficiency of Riemann Integral. In general, there are at least three issues about Riemann Integral:
\begin{enumerate}
\item Riemann integration does not handle functions with many discontinuities.
\item Riemann integration does not handle unbounded functions.
\item Riemann integration does not work well with limits.
\end{enumerate}
\subsection{Problem with many discontinuities: Dirichlet function on $[0, 1]$}
\begin{example}
Define $f:[0,1] \rightarrow \mathbb{R}$ by:
$$
f(x) = 
\begin{cases}
1 & \mbox{if }\; x \in [0, 1]\cap\mathbb{Q}\\
0 & \mbox{if }\; x \in [0, 1]\setminus\mathbb{Q}
\end{cases}
$$
\end{example}

According to the definition, for the Riemann integral defined on $[a, b]$, define partition $P:= \left\{a_{0}, a_{1}, ...,a_{n-1}, a_{n}\right\}$ with $a = a_{0} < a_{1} < ... < a_{n-1} < a_{n} = b$. Partition $P$ gives us lower and upper Riemann sums:

\begin{align*}
&L(P, f):= \sum\limits^{n}_{i=1}m_{i}\cdot\Delta a_{i} = \sum\limits^{n}_{i=1}\inf_{\scriptscriptstyle a_{i-1}\leq x \leq a_{i}}{f(x)\cdot(a_{i}-a_{i-1})}\\
&U(P, f):= \sum\limits^{n}_{i=1}M_{i}\cdot\Delta a_{i} = \sum\limits^{n}_{i=1}\sup_{\scriptscriptstyle a_{i-1}\leq x \leq a_{i}}{f(x)\cdot(a_{i}-a_{i-1})}
\end{align*}

The Riemann integral exists if and only if the lower integral equals to the upper integral of $f$ on $[a, b]$, i.e.:

$$
\sup\left\{L(P, f):\mbox{ $P$ is a partition on }[a, b]\right\} = \inf\left\{U(P, f):\mbox{ $P$ is a partition on }[a, b]\right\}
$$

For any partition $P$ and for any interval $[a_{i-1}, a_{i}]$ it gives, since both rational numbers and irrational numbers exist in any interval, thus we have:
\begin{align*}
&\inf_{\scriptscriptstyle a_{i-1}\leq x \leq a_{i}}{f(x)\cdot(a_{i}-a_{i-1})} = 0\\
&\sup_{\scriptscriptstyle a_{i-1}\leq x \leq a_{i}}{f(x)\cdot(a_{i}-a_{i-1})} = a_{i}-a_{i-1}
\end{align*}

Then: $L(P, f) = 0$ and $U(P,f) = \sum\limits^{n}_{i=1}(a_{i}-a_{i-1}) = 1$ for any partition $P$, and hence the upper integral does not equal to the lower integral, since:
\begin{align*}
&\sup\left\{L(P, f):\mbox{ $P$ is a partition on }[a, b]\right\} = 0\\
&\inf\left\{U(P, f):\mbox{ $P$ is a partition on }[a, b]\right\} = 1
\end{align*}

Therefore we can conclude that $f$ is not Riemann integrable, but since there are far more irrational number than rationals numbers, $f$ should, in some sense, have integral 0. However, the Riemann integral of $f$ is not defined.

\subsection{Problem with unbounded functions}
\begin{example}
Define $f: [0, 1] \rightarrow \mathbb{R}$ by:
$$
f(x)= 
\begin{cases}
\frac{1}{\sqrt{x}} & \text{if }\; 0 < x \leq 1\\
0 & \text{if }\; x = 0
\end{cases}
$$
\end{example}

For any partition $P:=  \left\{a_{0}, a_{1}, ..., a_{n}\right\}$ on [0, 1], for the first interval $[a_{0}, a_{1}] = [0, a_{1}]$, we always have:\\
$$
\sup_{\scriptscriptstyle a_{0}= 0 \leq x \leq a_{1}}f(x) \cdot a_{1} = \infty
$$

and then for any partition $P$ of $[0, 1]$:
$$
U(P, f):= \sum\limits^{n}_{i=1}\sup_{\scriptscriptstyle a_{i-1}\leq x \leq a_{i}}{f(x)\cdot(a_{i}-a_{i-1})} = \infty
$$ 

However, we should consider the area under the graph of $f$ to be $2$, not $\infty$, because $\lim\limits_{a\rightarrow0^{+}}\int^{1}_{a}f(x) dx = \lim\limits_{a\rightarrow 0^{+}}(2-2\sqrt{a}) = 2$. Calculus courses deal with this example by defining $\int^{1}_{0}\frac{1}{\sqrt{x}} dx = \lim\limits_{a\rightarrow 0^{+}}\int^{1}_{a}\frac{1}{\sqrt{x}}dx$.That is, we can use the approach in Calculus to calculating Riemann integrals for these functions. 

\begin{example}
Using the same approach above, and
$$
f(x) = \frac{1}{\sqrt{x}} + \frac{1}{\sqrt{1-x}}
$$

Then we would define $\int^{1}_{0} f(x) dx$ to be:\
$$
\int^{1}_{0} f(x) dx = \lim\limits_{a\rightarrow 0^{+}}\int^{\frac{1}{2}}_{a} f(x) dx + \lim\limits_{b\rightarrow 1^{-}}\int^{b}_{\frac{1}{2}} f(x) dx
$$ 

However, this technique (i.e. taking Riemann integrals over subdomains and then taking limits can fail with more complicated functions, as shown in the following example.
\end{example}

\begin{example}
Let $\left\{r_{k}\right\}_{k \in \mathbb{Z}^{+}}$, be a sequence which includes each rational number in $(0, 1)$ exactly once. For $k \in \mathbb{Z}^{+}$, define $f_{k}: [0, 1] \rightarrow \mathbb{R}$ by:
$$
f_{k}(x) = 
\begin{cases}
\frac{1}{\sqrt{x-r_{k}}} & \mbox{ if } \; x > r_{k}\\
0 & \mbox{ if } x \; \leq r_{k}
\end{cases}
$$

Define $f: [0, 1] \rightarrow [0, \infty]$ by:
$$
f(x) = \sum\limits^{\infty}_{k=1}\frac{f_{k}(x)}{2^{k}}
$$
\end{example}

For every non-empty open sub-interval $(a, b) \subset [0, 1]$, it must contains a rational number $r_{h} \in (a, b)$, then $a < r_{h} < b$. Note that $r_{h}$ is also belongs to the sequence ${r_{k}}_{k \in \mathbb{Z}^{+}}$ because $r_{h}$ is a rational number in $[0, 1]$. Hence for that $r_{h}$, we have:

$$
f_{h}(x) =
\begin{cases}
\frac{1}{\sqrt{x-r_{h}}} & \mbox{ if } \; x > r_{h}\\
0 & \mbox{ if } \; x \leq r_{h}
\end{cases}
$$

For $x\rightarrow r^{+}_{h}, f_{h}(x)\rightarrow \infty$, meaning $f_{h}(x)$ is unbounded. Thus, function $f$ is unbounded on every its non-empty open sub-interval. Therefore, the Riemann integral of $f$ is undefined on every its non-empty open sub-interval.

However, the area under the graph of each $f_{k}$ is less than 2:
$$
\int^{1}_{0} f_{k}(x) dx =\lim\limits_{\scriptscriptstyle t \rightarrow 1^{-}, s \rightarrow r^{+}_{k}}\int^{t}_{s} \frac{1}{\sqrt{x-r_{k}}} dx \;=\; 2\sqrt{1-r_{k}} \;< \; 2
$$

Then the formula defining $f$: $f(x) = \sum\limits^{\infty}_{k=1}\frac{f_{k}(x)}{2^{k}}$ shows that we should expect the area under the graph of $f$ to be less than 2, instead of being undefined. 

\subsection{Problem with point-wise limits}
\begin{example}
Let $\left\{r_{k}\right\}_{k \in \mathbb{Z}^{+}}$, be a sequence which includes each rational number in $[0, 1]$ exactly once. For $k \in \mathbb{Z}^{+}$, define $f_{k}: [0, 1] \rightarrow \mathbb{R}$ by:
$$
f_{k}(x) = 
\begin{cases}
1 &  \mbox{ if } \; x \in \left\{r_{1},..., r_{k}\right\}\\
0 &  \mbox{ otherwise}
\end{cases}
$$
\end{example}

\begin{remark}
Note that the difference between each function $f_{k}$ in this example and function $f$ mentioned in \textbf{Example 1} is that, $f_{k}$ is a function which takes $1$ on finite points $r_{1},...,r_{k}$, while $f$ is a function which takes $1$ on all rational numbers on $[0, 1]$.
\end{remark}

For any $k \in \mathbb{Z}^{+}$, function $f_{k}$ is Riemann integrable: For lower integral, $L(P, f_{k}) = 0$ for any partition $P$ on $[0, 1]$, since $\inf\limits_{\scriptscriptstyle a_{i-1} \leq x \leq a_{i}}f_{k}(x) = 0$ for all $a_{i-1}, a_{i} \in P$. Hence the lower integral of $f_{k}$ is $0$. 

For upper integral and the pair $a_{m-1}, a_{m}$ such that $r_{k} \in [a_{m-1}, a_{m}]$, we have:
$$
\sup\limits_{\scriptscriptstyle a_{m-1} \leq x \leq a_{m}}f_{k}(x)\cdot(a_{m}-a_{m-1}) = a_{m}-a_{m-1},
$$

and $f_{k} = 0$ on other intervals which do not include $r_{k}$, thus $\inf(a_{m}-a_{m-1}) = 0$ by refining partition $P$. Hence the upper integral of $f_{k}$ is also $0$. Because both lower and upper integral are 0, $f_{k}$ is Riemann integrable with $\int^{1}_{0}f_{k}(x) dx = 0$.

Define $f: [0, 1] \rightarrow \mathbb{R}$ by:
$$
f(x) = 
\begin{cases}
1 & \; \mbox{ if x is rational}\\
0 & \; \mbox{ if x is irrational}
\end{cases}
$$

Clearly $\lim\limits_{k \rightarrow \infty}f_{k}(x) = f(x)$ for each $x \in [0, 1]$.

However, $f$ is not Riemann integrable (\textbf{Example 1}) even though $f$ is the point-wise limit of a sequence of integrable functions bounded by 1.\\
~\\

\begin{proposition}
There does not exist a function $\mu$ with all the following properties:\\

(a) $\mu$ is a function from the set of all subsets of $\mathbb{R}$ to $[0, \infty]$.\\

(b) $\mu(I) = l(I)$ for every open interval $I$ of $\mathbb{R}$.\\

(c) $\mu\left(\bigcup\limits^{\infty}_{k=1}A_{k}\right) = \sum\limits^{\infty}_{k=1}\mu(A_{k})$ for every disjoint sequence $A_{1}, A_{2},...$ of subsets of $\mathbb{R}$.\\

(d) $\mu(A + t) = \mu(A)$ for every $A \subset \mathbb{R}$ and every $t \in \mathbb{R}$.
\end{proposition}

\begin{proof}

Suppose there exists a function $\mu$ with all the properties listed above. Observe that $\mu(\emptyset) = 0$ by (b), because empty set is an open interval with length $0$. We are now showing that $\mu$ has all the properties of outer measure that were used in the \textbf{disproof of additivity of outer measure}.

If $A \subset B \subset \mathbb{R}$, then $\mu(A) \leq \mu(B)$ by (c), because we can rewrite $B$ as the unions of disjoint sequence: $A, B \setminus A, \emptyset, \emptyset,...$, thus:

$$
\mu(B) = \mu(A) + \mu(B \setminus A) + 0 + 0 + ... = \mu(A) + \mu(B \setminus A) \geq \mu(A)
$$

If $a, b \in \mathbb{R}$ with $a < b$, then $(a, b) \subset [a, b] \subset (a-\epsilon, b+\epsilon)$ for every $\epsilon > 0$. Thus $b-a \leq \mu([a, b]) \leq b-a+2\epsilon$ for every $\epsilon > 0$. Hence $\mu([a, b]) = b-a$.

If $A_{1}, A_{2}, ...$ is a sequence of subsets of $\mathbb{R}$, then $A_{1}, A_{2}\setminus A_{1}, A_{3}\setminus (A_{1} \cup A_{2}), ...$ is a disjoint sequence of subsets of $\mathbb{R}$ whose union is $\bigcup\limits^{\infty}_{k=1}A_{k}$, thus:

\begin{align*}
\mu\left(\bigcup\limits^{\infty}_{k=1}A_{k}\right) &= \mu(A_{1} \cup (A_{2}\setminus A_{1}) \cup (A_{3}\setminus (A_{1} \cup A_{2})) \cup ...)\\
&= \mu(A_{1}) + \mu(A_{2}\setminus A_{1}) + \mu(A_{3}\setminus (A_{1} \cup A_{2}))) + ...\\
&\leq \sum\limits^{\infty}_{k=1}\mu(A_{k})
\end{align*}

Until now, we have shown that $\mu$ has all the properties of outer measure that were used in the \textbf{disproof of additivity of outer measure}. However, from the disproof we know that there exist disjoint subsets $A, B$ of $\mathbb{R}$ such that $\mu(A \cup B) \neq \mu(A) + \mu(B)$. Thus the disjoint seqeunce $A, B, \emptyset, \emptyset, ...$ does not satisfy the countable additivity required by (c). This contradiction completes the proof.
\end{proof}
\newpage
The last-page result shows that we need to give up one of the desirable properties in our goal of extending the notion of size from intervals to more general subsets of $\mathbb{R}$. 

We cannot give up (b) because the size of an interval needs to be its length. 

We cannot give up (c) because countable additivity is needed to prove theorems about limits. 

We cannot give up (d) because a size that is not translation invariant does not satisfy our intuitive notion of size as a generalization of length.

Thus we are forced to relax the requirement in (a) that the size is defined for all subsets of $\mathbb{R}$. Experience shows that to have a viable theory that allows for taking
limits, the collection of subsets for which the size is defined should be closed under
complementation and closed under countable unions. Therefore, we have the definition of $\sigma-\mbox{field}$.\\
~\\
\section{Theorems of Lebesgue Measure}

\begin{theorem}
Any null set N is measurable and $m(N) = 0$.\\
i.e. For $\forall \mbox{ null set } N: \forall A \subset \mathbb{R}: m^{*}(A) = m^{*}(A \cap N) + m^{*}(A \cap N^{c})$
\end{theorem}

\begin{proof}
Firstly, since $(A \cap N) \subset N$, by theorem of outer measure proved before, we have $0 \leq m^{*}(A \cap N) \leq m^{*}(N) = 0$, then we have $m^{*}(A \cap N) = 0$. Now we prove $m^{*}(A \cap N^{c}) = m^{*}(A)$.

Since $(A \cap N^{c}) \subset A$, similarly, $m^{*}(A \cap N^{c}) \leq m^{*}(A)$. Let's suppose that $m^{*}(A \cap N^{c}) < m^{*}(A)$. By sub-additivity of outer measure, $m^{*}(A) = m^{*}\big((A \cap N^{c})\cup N\big) \leq m^{*}(A \cap N^{c}) + m^{*}(N)$ , which means $m^{*}(A \cap N^{c}) < m^{*}(A \cap N^{c}) + m^{*}(N) \iff m^{*}(N) > 0$, which is a contradiction to the Theorem(i) proved before. Hence, $m^{*}(A \cap N^{c}) = m^{*}(A)$.

Therefore, we can see that any null set N is Lebesgue-measurable, and then its Lebesgue measure equals to its outer measure: $m(N) = m^{*}(N) = 0$.
\end{proof}
~\\
\begin{theorem}{\mbox{}}
Any interval $\big([a, b], (a, b) and etc.\big)$ is measurable, and for $-\infty < a \leq b < \infty$: $m\big([a, b]\big) = m\big((a, b)\big) = b - a$.

Noted that this theorem is very similar and may be equivalent to another theorem (but I haven't prove it yet), that is, if $A$ and $G$ are disjoint and $G$ is open, then $m^{*}(A \cup G) = m^{*}(A) + m^{*}(G)$.
\end{theorem}



\begin{proof}
~\
With out loss of generality, let $E = (a, b]$. $E$ is measurable $\iff$: 
$$
\forall A \subset \mathbb{R}: m^{*}(A) = m^{*}(A \cap E) + m^{*}(A \cap E^{c})
$$

By sub-additivity of outer measure, we have:
$$
m^{*}(A) \leq m^{*}(A \cap E) + m^{*}(A \cap E^{c})
$$

Now prove that the other side of above inequality holds, the LHS:
\begin{align*}
m^{*}(A \cap E) + m^{*}(A \cap E^{c}) & = m^{*}(A \cap (a, b]) + m^{*}(A \cap ((-\infty, a] \cup (b, \infty)))\\
& = m^{*}(A \cap (a, b]) + m^{*}((A \cap (-\infty, a]) \cup (A \cap (b, \infty)))\\
\bigstar & \leq m^{*}(A \cap (a, b]) + m^{*}(A \cap (-\infty, a]) + m^{*}(A \cap (b, \infty))\\
& = m^{*}(A \cap (a, b)) + m^{*}(A \cap (-\infty, a)) + m^{*}(A \cap (b, \infty))
\end{align*}

Note that $\left\{ a, b \right\}$ are also excluded in step $\bigstar$, because a few single points don't affect the outer measure. Since $m^{*}(A)$ is the outer measure of set $A$, by definition,
$$
\forall \epsilon > 0: \exists \left\{ L_{n} \right\}^{\infty}_{n=1}, L_{n} \mbox{ are open intervals, }, A \subset \bigcup\limits^{\infty}_{n=1}L_{n}: \sum\limits^{\infty}_{n=1} l(L_{n}) < m^{*}(A) + \epsilon
$$

Notice that ``open intervals" are used to construct a cover to $A$, this is equivalent to the construction of ``intervals", because outer measure is the infimum of such cover, for the same reason as above, wouldn't be changed by a few points.\\

Then for any $\epsilon > 0$ and the cover $\left\{ L_{n} \right\}^{\infty}_{n=1}$ that exists correspondingly, let:\\

$\qquad I_{n} = L_{n} \cap (-\infty, a)$, $\qquad J_{n} = L_{n} \cap (a, b)$, $\qquad K_{n} = L_{n} \cap (b, \infty)$

We thus have:
\begin{align*}
& (A \cap (-\infty, a)) \subset \left( \left( \bigcup^{\infty}_{n=1}L_{n} \right) \bigcap (-\infty, a) \right) = \bigcup^{\infty}_{n=1}(L_{n} \cap (-\infty, a)) = \bigcup^{\infty}_{n=1}I_{n}\\
& (A \cap (a, b)) \subset \left( \left( \bigcup^{\infty}_{n=1}L_{n} \right) \bigcap (a, b) \right) = \bigcup^{\infty}_{n=1}(L_{n} \cap (a, b)) = \bigcup^{\infty}_{n=1}J_{n}\\\
& (A \cap (b, \infty)) \subset \left( \left( \bigcup^{\infty}_{n=1}L_{n} \right) \bigcap (b, \infty) \right) = \bigcup^{\infty}_{n=1}(L_{n} \cap (b, \infty)) = \bigcup^{\infty}_{n=1}K_{n}\\
& \implies (A \cap (-\infty, a)) \subset \bigcup^{\infty}_{n=1}I_{n}, \quad (A \cap (a, b)) \subset \bigcup^{\infty}_{n=1}J_{n}, \quad (A \cap (b, \infty)) \subset \bigcup^{\infty}_{n=1}K_{n}
\end{align*}

Moreover, since intervals on $\mathbb{R}$ are closed under intersection, i.e. the intersection of any two intervals is a interval. In special, the intersection of any two open intervals is an open interval. Thus, for any $n \in \mathbb{N}^{+}$, $I_{n}, J_{n}, K_{n}$ are pair-wise disjoint open intervals, which implies that $m^{*}(I_{n} \cup J_{n} \cup K_{n}) = l(I_{n}) + l(J_{n}) + l(K_{n})$. We also have:
$$
I_{n} \cup J_{n} \cup K_{n} = L_{n} \cap (\mathbb{R} \setminus \left\{ a, b \right\}) = L_{n} \setminus \left\{ a, b \right\} \implies m^{*}(I_{n} \cup J_{n} \cup K_{n}) = l(L_{n})
$$ 
~\\
Then $l(L_{n}) = l(I_{n}) + l(J_{n}) + l(K_{n})$. Since: 
\begin{align*}
& m^{*}(A \cap (a, b)) + m^{*}(A \cap (-\infty, a)) + m^{*}(A \cup (b, \infty))\\
& \leq m^{*}(\bigcup^{\infty}_{n=1}I_{n}) + m^{*}(\bigcup^{\infty}_{n=1}J_{n}) + m^{*}(\bigcup^{\infty}_{n=1}K_{n})\\
& \leq \sum\limits^{\infty}_{n=1}m^{*}(I_{n}) + \sum\limits^{\infty}_{n=1}m^{*}(J_{n}) + \sum\limits^{\infty}_{n=1}m^{*}(K_{n})\\
& = \sum\limits^{\infty}_{n=1}\left( l(I_{n}) + l(J_{n}) + l(K_{n}) \right) = \sum\limits^{\infty}_{n=1}l(L_{n})
\end{align*}

Therefore, for any $\epsilon > 0$, and for that cover $\left\{ L_{n} \right\}^{\infty}_{n=1}$ exists correspondingly, we have the following inequality holds:
$$
m^{*}(A \cap (a, b)) + m^{*}(A \cap (-\infty, a)) + m^{*}(A \cap (b, \infty)) \leq \sum\limits^{\infty}_{n=1}l(L_{n}) < m^{*}(A) + \epsilon
$$

Along with another property of infimum, that $m^{*}(A) \leq \sum\limits^{\infty}_{n=1}l(L_{n})$ for any cover $\left\{ L_{n} \right\}^{\infty}_{n=1}$, 
$$
m^{*}(A) \leq \sum\limits^{\infty}_{n=1}l(L_{n}) < m^{*}(A)
$$

By letting $\epsilon \rightarrow 0$, we have:
\begin{align*}
m^{*}(A) = \sum\limits^{\infty}_{n=1}l(L_{n}) & \geq m^{*}(A \cap (a, b)) + m^{*}(A \cap (-\infty, a)) + m^{*}(A \cup (b, \infty))\\
& \geq m^{*}(A \cap (a, b]) + m^{*}(A \cap ((-\infty, a] \cup (b, \infty)))
\end{align*}

Along with the conclusion drawed by the sub-additivity of outer measure, we have both: 
\begin{align*}
m^{*}(A) \geq m^{*}(A \cap (a, b]) + m^{*}(A \cap ((-\infty, a] \cup (b, \infty)))\\
m^{*}(A) \leq m^{*}(A \cap (a, b]) + m^{*}(A \cap ((-\infty, a] \cup (b, \infty)))
\end{align*}

Therefore, $m^{*}(A) = m^{*}(A \cap (a, b]) + m^{*}(A \cap ((-\infty, a] \cup (b, \infty)))$. It shows that for any interval $E \subset \mathbb{R}$, $E$ is measurable, and its Lebesgue-measure equals to its outer measure, i.e. the length of the interval.
\end{proof}
~\\
\begin{theorem}
$\mathbb{R} \in \mathcal{M}$; of course $m(\mathbb{R}) = \infty$.
\end{theorem}

\begin{proof}
For any $A \subset \mathbb{R}$, we have $m^{*}(A \cap \mathbb{R}) = m^{*}(A), m^{*}(A \cap \mathbb{R}^{c}) = m^{*}(\emptyset) = 0$, hence $m^{c}(A) = m^{*}(A \cap \mathbb{R}) + m^{*}(A \cap \mathbb{R})^{c}$. Then $\mathbb{R}$ is Lebesgue-measurable with $m(\mathbb{R}) = \infty$.
\end{proof}
~\\
\begin{theorem}[\textbf{Lebesgue-measure is Closed under Complement}]{\mbox{}}
If $E \in \mathcal{M}$, then $E^{c} \in \mathcal{M}$.
\end{theorem}

\begin{proof}
If  $E \in \mathcal{M}$, i.e. $E$ is Lebesgue-measurable, then:
$$
\forall A \subset \mathbb{R}: m^{*}(A) = m^{*}(A \cap E) + m^{*}(A \cap E^{c})
$$

Since $(E^{c})^{c} = E$, then for any $A \subset \mathbb{R}$,
\begin{align*}
m^{*}(A \cap E^{c}) + m^{*}\left(A \cap (E^{c})^{c}\right) & = m^{*}(A \cap E) + m^{*}(A \cap E^{c})\\
& = m^{*}(A)
\end{align*}

Hence $E^{c}$ is also Lebesgue-measurable. The proof shows that $\mathcal{M}$ is closed under complement.
\end{proof}
~\\
\begin{theorem}[\textbf{Lebesgue-Measure is Closed under Countable Unions}]{\mbox{}}

If $E_{n} \in \mathcal{M}$ for all $n = 1, 2, 3, ...$, then $\bigcup\limits^{\infty}_{n=1}E_{n} \in \mathcal{M}$.
\end{theorem}

\begin{proof}
Proof by induction:

For $n = 1, 2$, that is, prove $E_{1} \cup E_{2}$ is Lebesgue-measurable if both $E_{1}$ and $E_{2}$ are Lebesgue-measurable, i.e.,
$$
\forall A \subset \mathbb{R}: m^{*}(A) = m^{*}\left((A \cap (E_{1} \cup E_{2})\right) + m^{*}\left( A \cap (E_{1}, E_{2})^{c} \right)
$$

Using the technique that $M \cup N = (M \setminus N) \cup (N \setminus M) \cup (M \cap N)$. For $A \subset \mathbb{R}$, the RHS:

\begin{align*}
m^{*} & \left(A \cap (E_{1} \cup E_{2})\right) + m^{*}\left( A \cap (E_{1} \cup E_{2})^{c} \right)\\
& = m^{*}\left(A \cap (E_{1} \cup E_{2})\right) + m^{*}\left( A \setminus (E_{1} \cup E_{2}) \right)\\
& = m^{*} \left(A \cap \left((E_{1} \setminus E_{2}) \cup (E_{2} \setminus E_{1}) \cup (E_{1} \cap E_{2})\right)\right) + m^{*}\left( A \setminus (E_{1} \cup E_{2}) \right)\\
& = m^{*} \left((A \cap (E_{1} \setminus E_{2})) \cup (A \cap (E_{2} \setminus E_{1})) \cup (A \cap E_{1} \cap E_{2})\right) + m^{*}\left( A \setminus (E_{1} \cup E_{2}) \right)\\
& \leq m^{*} \left(A \cap (E_{1} \setminus E_{2})\right) + m^{*}\left(A \cap (E_{2} \setminus E_{1})\right) + m^{*}\left(A \cap E_{1} \cap E_{2}\right) + \left( A \setminus (E_{1} \cup E_{2}) \right)\\
& = \left[ m^{*}\left((A \cap E_{1}) \cap E_{2}\right) + m^{*}\left((A \cap E_{1}) \setminus E_{2} \right)\right]\\
& \qquad \qquad \qquad \qquad \qquad + \left[ m^{*}\left((A \setminus E_{1}) \cap E_{2}\right) + m^{*}\left((A \setminus E_{1}) \setminus E_{2} \right)\right]
\end{align*}
~\\

Since $E_{2}$ is Lebesgue-measurable, by definition: 
$$
\forall F \subset \mathbb{R}: m^{*}(F) = m^{*}(F \cap E_{2}) + m^{*}(F \cap E^{c}_{2})
$$

For $F = A \cap E_{1}$ where $A$ is determined, we have:
\begin{align*}
m^{*}(A \cap E_{1}) & = m^{*}\left((A \cap E_{1}) \cap E_{2}\right) + m^{*}\left((A \cap E_{1}) \setminus E_{2}\right)\\
& = m^{*}\left((A \cap E_{1}) \cap E_{2}\right) + m^{*}\left((A \cap E_{1}) \cap E^{c}_{2}\right)
\end{align*}

For $F = A \setminus E_{1}$ where $A$ is determined, we have:
\begin{align*}
m^{*}(A \setminus E_{1}) & = m^{*}\left((A \setminus E_{1}) \cap E_{2}\right) + m^{*}\left((A \setminus E_{1}) \setminus E_{2}\right)\\
& = m^{*}\left((A \setminus E_{1}) \cap E_{2}\right) + m^{*}\left((A \setminus E_{1}) \cap E^{c}_{2}\right)
\end{align*}

Therefore, continued from the above, that:
\begin{align*}
m^{*} & \left(A \cap (E_{1} \cup E_{2})\right) + m^{*}\left( A \cap (E_{1} \cup E_{2})^{c} \right)\\
& \leq \left[ m^{*}\left((A \cap E_{1}) \cap E_{2}\right) + m^{*}\left((A \cap E_{1}) \setminus E_{2} \right)\right]\\
& \qquad \qquad \qquad \qquad \qquad + \left[ m^{*}\left((A \setminus E_{1}) \cap E_{2}\right) + m^{*}\left((A \setminus E_{1}) \setminus E_{2} \right)\right]\\
& = m^{*}\left(A \cap E_{1} \right) + m^{*}\left(A \setminus E_{1} \right) = m^{*}\left(A \cap E_{1} \right) + m^{*}\left(A \cap E^{c}_{1} \right)\\
& = m^{*}\left( A \right)
\end{align*}

The last equality holds because $E_{1}$ is also Lebesgue-measurable. Therefore, we have: 
$$
m^{*} \left(A \cap (E_{1} \cup E_{2})\right) + m^{*}\left( A \cap (E_{1} \cup E_{2})^{c} \right) \leq m^{*}\left( A \right)
$$

Combined with conclusion directly drawed by the sub-additivity of outer measure that: $ m^{*} \left(A \cap (E_{1} \cup E_{2})\right) + m^{*}\left( A \cap (E_{1} \cup E_{2})^{c} \right) \geq m^{*}\left( A \right)$, we therefore have: 
$$
m^{*} \left(A \cap (E_{1} \cup E_{2})\right) + m^{*}\left( A \cap (E_{1} \cup E_{2})^{c} \right) = m^{*}\left( A \right)
$$

By induction, for general case that $E_{1}, E_{2}, ...$ are all Lebesgue-measurable, then $E_{1} \cup E_{2}$ is Lebesgue-measurable, then $E_{1} \cup E_{2} \cup E_{3}$ is Lebesgue-measurable, ... The proof hence to be completed and shows that $\mathcal{M}$ is closed under countable unions.
\end{proof}

\begin{remark}{\mbox{}}
Combining the above three theorems, $\mathcal{M}$ is therefore a $\sigma$-field. The Lebesgue-measure is a function with mapping relation: $\mathcal{M} \rightarrow [0, \infty]$. Besides, it is also countably additive as the following theorem.
\end{remark}
~\\
\begin{theorem}[\textbf{Countable Additivity of Lebesgure-Measure}]
~\

If $E_{n} \in \mathcal{M}$ for $n = 1, 2 ,...$, and $E_{j} \cap E_{k} = \emptyset$ for $j \neq k$, then:
$$
m\left( \bigcup\limits^{\infty}_{n=1}E_{n} \right) = \sum\limits^{\infty}_{n=1}m\left( E_{n} \right)
$$
\end{theorem}

\begin{proof}
Let $E_{n}$ be Lebesgue-measurable sets for $\forall n \in \mathbb{N}^{+}$, becuase Lebesgue-measure is closed under countable unions as proved above, we have: $\bigcup\limits^{\infty}_{n=1} E_{n}$ is Lebesgue-measurable, with Lebesgue measure equals to its outer measure, i.e.
$$
m\left(\bigcup\limits^{\infty}_{n=1} E_{n}\right) = m^{*}\left(\bigcup\limits^{\infty}_{n=1} E_{n}\right)
$$ 
By sub-additivity of outer measure, we have:
$$
m^{*}\left(\bigcup\limits^{\infty}_{n=1} E_{n}\right) \leq \sum\limits^{\infty}_{n=1} m^{*}(E_{n}) = \sum\limits^{\infty}_{n=1} m(E_{n})
$$

The last equality holds because $E_{n}$ is Lebesgue-measurable.

For any $E_{n}$, by definition, for any $\epsilon > 0$, there exists a cover $\left\{ I^{n}_{m} \right\}^{\infty}_{m=1}$ where $E_{n} \subset \bigcup\limits^{\infty}_{m=1} I^{n}_{m}$ and $I^{n}_{m}$ are open intervals, such that: $\sum\limits^{\infty}_{m=1}l(I^{n}_{m}) < m^{*}\left(E_{n}\right) + \epsilon$.
\end{proof}

\newpage
\begin{corollary}[\textbf{Equivalences for being a Lebesgue-Measurable Set}]
\normalfont
~\\

Suppose $E \subset \mathbb{R}$. Then the following are equivalent.\\
\begin{enumerate}
\item $E$ is Lebesgue-measurable.\\
\item For any $\epsilon > 0$: there exists a closed set $F \subset E$ with $m^{*}(E \setminus F) < \epsilon$.\\
\item There exist closed sets $F_{1}, F_{2}, ...$ contained in $E$ such that:\\
$m^{*}\left( E \setminus\bigcup\limits^{\infty}_{k=1} F_{k} \right) = 0.$\\
\item There exists a Borel set $B \subset E$ such that $m^{*}\left( E \setminus B \right) = 0$.\\
\item For any $\epsilon > 0$: there exists an open set $G \supset E$ such that: \\
$m^{*}\left( \left( \bigcap\limits^{\infty}_{k=1} G_{k} \right) \setminus E \right) = 0$.
\end{enumerate}
\end{corollary}
~\\
\begin{definition}[\textbf{Measurable Space; Measurable Set; Measure}]{\mbox{}}
\begin{enumerate}
\normalfont
\item[$\bullet$] A measurable space is on ordered pair $\left( \Omega, \mathcal{F} \right)$, where $\Omega$ is a set and $\mathcal{F}$ is a $\sigma$-field on $\Omega$.\\

\item[$\bullet$] An element of $\mathcal{F}$ is called an $\mathcal{F}$-measurable set, or just a measurable set if $\mathcal{F}$ is clear from the context.\\

\item[$\bullet$] A real-valued function $\mathbf{\mu}: \mathcal{F} \rightarrow [0, \infty]$ is called a measure, if:
\begin{enumerate}
\item $\mu\left( \emptyset \right) = 0$
\item For any sequence of disjoint subsets $E_{n} \in \mathcal{F}, n = 1, 2, ...$, \\
we have: $\mu\left( \bigcup\limits^{\infty}_{n=1} E_{n} \right) = \sum\limits^{\infty}_{n=1}\mu\left( E_{n} \right)$
\end{enumerate}
$\left( \Omega, \mathcal{F}, \mu \right)$ is called a \underline{measure space}; $\left( \Omega, \mathcal{F} \right)$ is called a \underline{measurable space}.
\end{enumerate}
\end{definition}
~\\
\begin{remark}
The \textbf{Borel $\mathbf{\sigma}$-field} is the minimal $\sigma$-field on $\Omega$ containing all open sets. When $\Omega = \mathbb{R}$ specifically, $\mathbf{B}$ (or $\mathcal{B}(\mathbb{R})$ specifically) is also the minimal $\sigma$-field containing all open intervals. This is because any open set can be written as the unions of open intervals, and  $\sigma$-field is closed under countable unions. 
~\\

Recall that $\mathcal{M}$ is the collection of all Lebesgue-measurable set on $\mathbb{R}$. Since $\mathcal{M}$ is a $\sigma$-field including all open intervals, $\mathcal{B}$ is the smallest $\sigma$-field generated by all open intervals, therefore $\mathcal{B} \subset \mathcal{M}$, or for any Borel set $B$, $B \in \mathcal{M}$.

\end{remark}
\newpage
\begin{definition}[\textbf{Complete Measurable Space and Completion}]{\mbox{}}

A measure space $\left(\Omega, \mathcal{F}, \mu \right)$ is \textbf{complete} in which any subset of any null set is $\mu$-measurable, i.e. $\mbox{for: }\forall G \in \mathcal{F}, \mu(G) = 0: \forall N \subset G: N \in \mathcal{F}$. $\mathcal{F}$ is also called a \textbf{complete measurable space}.
~\\

Given any measure space $\left( \Omega, \mathcal{G}, \mu^{*} \right)$ (possibly imcomplete), there is an extension of this measure space that is complete. The smallest such extension is called a \textbf{completion} of the measure space $\left( \Omega, \mathcal{G}, \mu^{*} \right)$. $\mu$ is the extension of $\mu^{*}$, which is given by $\mu(E):= \inf{\left\{ \mu^{*}(E): E \subset D \in \mathcal{F}\right\}}$.

Alternatively, a \textbf{completion} $\mathcal{F}$ of a $\sigma$-field $\mathcal{G}$, relative to a given measure $\mu$ (or complete measure space $\left( \Omega, \mathcal{G}, \mu^{*} \right)$) is the minimal $\sigma$-field containing $\mathcal{G}$, such that any subset N of any null set $G \in \mathcal{G}$ is in $\mathcal{G}$, i.e. $\forall G \in \mathcal{G}, \mu^{*}(G)= 0: \forall N \subset G: N \in \mathcal{G}$. 
~\\

Notice the difference between above two definitions, the former defines completion as a measure space, while the latter defines completion as a $\sigma$-field. By no means, $\mathcal{M}$ is the completion of Borel $\sigma$-field $\mathcal{B}$ relative to the Lebesgue measure, and the Lebesgue measure is well-defined on Borel $\sigma$-field $\mathcal{B}$
\end{definition}
~\\

\begin{example}{\mbox{}}
Let $B \in \mathcal{M}$ be a fixed measurable set with $m(B) > 0$. Define $M_{B}:= \left\{A \cap B: A \in \mathcal{M}\right\}$. For $E \in \mathcal{M}_{B}$, define $m_{B}(E) = m(E)$. $\mathcal{M}_{B}$ is the collection of the Lebesgue-measurable subsets of $B$, and $m_{B}$ is the Lebesgue measure on $\mathcal{M}_{B}$, the \textbf{restriction} (or \textbf{trace}) of the Lebesgue measure $m$ to $B$. Now $(B, \mathcal{M}_{B}, m_{B})$ is a complete measure space.
~\\

First of all, $M_{B}$ is constructed by $A \cap B$, since $A$ and $B$ are Lebesgue-measurable, then $A \cap B$ is also Lebesgue-measurable, (i.e. any element in $\mathcal{M}_{B}$ is Lebesgue-measurable). Secondly, for any measurable set $E \subset B$, we have $E \cap B$ also measurable, then by contruction we have $E \cap B = E \in \mathcal{M}_{B}$, which shows that $\mathcal{M}_{B}$ contains all measurable subsets of $B$. Therefore, $\mathcal{M}_{B}$ contains and only contains all measurable subsets of $B$. Thirdly, for $G \in \mathcal{M}_{B}$ where $m_{B}(G) = 0$, we have $G \in \mathcal{M}$. Since $\mathcal{M}$ is complete, then for any $N \subset G$ we have $N \in \mathcal{M}$. Therefore $N \in \mathcal{M}_{B}$ because $N$ is a subset of $B$ and $\mathcal{M}_{B}$ contains all measurable subsets of $B$.
\end{example}


















\end{document} 